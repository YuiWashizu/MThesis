\chapter{外部トリガを用いた応答評価試験}
この章では,外部トリガを用いた粒子線に対する応答評価試験について述べる.

\section{外部トリガを用いた応答評価試験セットアップ}
RD53Aの量産にあたって,行われる品質試験では,外部トリガを用いた応答評価試験が行われる.そのため,実際に行われる試験環境に近いセットアップで試験を行なった.セットアップの写真を以下に示す.
実際に行われる試験環境はクーリングボックスと呼ばれる,温度が低温に維持された小さな箱の中で行う.それに合う応答評価試験セットアップを考えるにあたって,線源とシンチレータ,シンチレータの光信号を電気信号に変換するMPPCが乗ったソースホルダと,MPPCからのアナログ信号を波形整形し,LVDSのデジタル信号に変換するような基板の設計を行なった.

\subsection{ソースホルダの設計}
ソースホルダの外観を以下に示す.クーリングボックス内で使用することを想定し,コンパクトな作りになっている.これは,FreeCADというオープンソース汎用3D CADモデラで設計し,3Dプリンタを用いて作成した.

\subsection{MPPCからの信号を波形整形する回路}
MPPCからの信号を波形整形する回路は以下のようになっている.\\
大きく,電圧供給回路,反転増幅回路,コンパレータ回路,LVDS変換回路から構成されている.基板はKiCADというCERN開発のオープンソースプリント基板CADを用いて設計し,によって作成した.

\section{応答評価試験手順}
\section{応答評価試験結果}
\section{考察}
