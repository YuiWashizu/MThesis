\chapter{外部トリガを用いた応答評価試験}
この章では,外部トリガを用いた粒子線に対する応答評価試験について述べる.\ref{sec:extsetup}節で外部トリガでデータ取得をする際のセットアップ,\ref{sec:exthow}節で手順,\ref{sec:extconc}節で取得データ結果を示し,\ref{sec:selfsum}節で考察を行なっている.

\section{外部トリガを用いた応答評価試験セットアップ}
\label{sec:extsetup}
RD53Aの量産にあたって,行われる品質試験では,外部トリガを用いた応答評価試験が行われる.そのため,実際に行われる試験環境に近いセットアップで試験を行なった.セットアップの写真を以下に示す.MPPCには58$\mathrm{V}$印加した.SCCとFPGAボード,PCを用いてシステムを組む大枠は変わらずに,線源の配置とトリガに使用するものが異なっている.\par
実際に行われる試験は,クーリングボックスと呼ばれる,温度が低温に維持された小さな箱の中で行い,トリガには前章で述べたHitOR信号ではなく,シンチレータからの光信号を用いる.シンチレータは線源とモジュールの間に設置するため,なるべく$\beta$線を遮らないように,厚みは0.5$\mathrm{mm}$と非常に薄いものを使用した.
実際に行われる試験環境はクーリングボックスと呼ばれる,温度が低温に維持された小さな箱の中で行う.それに合う応答評価試験セットアップを考えるにあたって,線源とシンチレータ,シンチレータの光信号を電気信号に変換するMPPCが乗ったソースホルダと,MPPCからのアナログ信号を波形整形し,LVDSのデジタル信号に変換するような基板の設計を行なった.

\subsection{ソースホルダの設計}
ソースホルダの外観を以下に示す.クーリングボックス内で使用することを想定し,コンパクトな作りになっている.これは,FreeCADというオープンソース汎用3D CADモデラで設計し,3Dプリンタを用いて作成した.

\subsection{MPPCからの信号を波形整形する回路}
MPPCからの信号を波形整形する回路は以下のようになっている.\par
大きく,電圧供給回路,反転増幅回路,コンパレータ回路,LVDS変換回路から構成されている.基板はKiCADというCERN開発のオープンソースプリント基板CADを用いて設計・作成した.回路図を以下に示す.

\subsubsection*{回路の動作確認}
MPPCからの信号を正しく波形整形できているかをオシロスコープを用いて確認した.増幅されたMPPCからの信号がコンパレータによって,デジタル信号に変換され,TTL to LVDS変換のチップによって,差動信号に変換されている様子がわかる.

\subsubsection*{トリガシンチの閾値設定}
この回路では,信号を出力をコンパレータの閾値によって設定している.この閾値によって,MPPCからの信号が削られることなく伝達されているかを確認した.

\section{応答評価試験手順}
\label{sec:exthow}
\section{応答評価試験結果}
\label{sec:extconc}
\section{考察}
\label{sec:extsum}
