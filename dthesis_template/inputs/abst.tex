Large Hadoron Collider (LHC)は欧州原子核研究機構(CERN)に設置された陽子陽子衝突型の粒子加速器である.LHCの4つの衝突点のうちの1つの設置されたATLAS検出器を用いて,素粒子標準模型の精密測定および,それを超えた物理の探索を行なっている実験がATLAS実験である.LHCでは,2026年の開始を目指して高輝度計画・High Luminosity LHC (HL-LHC)計画が進められている.LHCのルミノシティを向上させることで,陽子中の大きなエネルギーを持つパートンの衝突を可能にし,重い粒子の探索を目的とした計画である.また,統計量が増えるために,超対称性などの様々な模型が予想する新粒子への感度の向上も期待されている.このHL-LHC計画に伴い,ATLAS検出器の内部飛跡検出器は,受ける放射線量の増加,検出器のヒット占有率の増加などに対応するために,Inner Tracker(ITk)と呼ばれるシリコン検出器への総入れ替えが予定されている.総入れ替えのために,内部に用いるピクセルセンサモジュールは量産され,品質性能基準を達成するために様々な試験にかけられる.\par
本論文では,この試験項目の1つである,粒子線に対する応答評価試験を取り扱った.この試験は2つの手法があり,モジュールに粒子が入射した時の信号のタイミングでデータ取得を行うセルフトリガを用いた手法と,シンチレータを用いて,シンチレータに粒子が入射した時の信号のタイミングでデータ取得を行う外部トリガを用いた手法がある.本論文では,応答評価試験方法の確立のために必要なファームウェア,ソフトウェアの開発と,2つの手法それぞれで応答評価試験を行い,その結果からそれぞれの手法の利点・欠点を比較した.





