\chapter{LHC ATLAS実験}

\section{LHC加速器}
Large Hadron Collider(LHC)はスイス、ジュネーブにある欧州原子核研究機構(CERN)の地下100m,周長26.7kmのリングで構成される円形加速器である.最大で14TeVの重心系エネルギーで陽子陽子衝突させることが可能な,世界最大の陽子陽子衝突型加速器である.新粒子の探索や,ヒッグス粒子やトップクォーク等の質量が大きい粒子を多く生成できるので,結合定数などの精密測定も行うことが可能である.\\
LHCは2010年から運転を開始し,7TeVから8TeVの重心系エネルギーで2012年まで稼働した.この期間をLHC Run-1と呼び,瞬間最高ルミノシティは$0.77\times10^{34} \mathrm{cm^{-2}s^{-1}}$であった.その後,2013年から2015年までのシャットダウン期間で加速器のアップグレードを行い,2015年からは重心系エネルギー$13\mathrm{TeV}$でLHC Run-2が始まり,2018年まで続いた.Run-2の3年間で得られた積分ルミノシティは約$150\mathrm{fb}^{-1}$であった.\\
LHCは2年間のシャットダウン期間を経て2021年から重心系エネルギー$14\mathrm{TeV}$のLHC Run-3を予定している.Run-3が約3年間運転したのち,シャットダウン期間を挟んで,High Luminosity LHC(HL-LHC)が開始する予定である.

\subsection{HL-LHC計画}


\section{ATLAS実験}
ATLAS実験はLHCの衝突点に設置されたATLAS検出器を用いて陽子陽子衝突から$\mathrm{TeV}$スケールまでの高エネルギー物理事象を探索する実験である.2012年には,LHC実験の1つであるCMS実験と共にヒッグス粒子を発見し,標準理論の完成お大きな役割を担った.世界最高エネルギーのLHCを使ったヒッグス粒子やトップクォークといった重い粒子の精密測定はATLAS実験の重要な目的の1つである.他にも超対称性粒子などの新粒子を発見することが特に大きな目的となっている.

\section{ATLAS検出器}
ATLAS検出器の全体図を示す.ATLAS検出器は直径25 $\mathrm{m}$,長さ44 $\mathrm{m}$の円筒形で,陽子同士の衝突点から生じる粒子を検出できる構造になっている.また,多数の検出器の複合体である,内側から層状に,内部飛跡検出器,電磁カロリメータ,ハドロンカロリメータ,ミューオン検出器の順に配置されている.これらの複数の検出器を組み合わせることにより,粒子の追跡と識別をすることが可能になる.以降,本論文に関する内部飛跡検出器について,詳しく述べる.

\subsection{現行の内部飛跡検出器}
現行の内部飛跡検出器は,半径1.15 $\mathrm{m}$,長さ7 $\mathrm{m}$の円筒形で,荷電粒子の飛跡を検出する.Insertable B-Layer(IBL),Pixel検出器,Semiconductor Tracker(SCT)とTransition Radiation Tracker(TRT)からなり,荷電粒子の飛跡を検出する,内部飛跡検出器の構造を以下に示す.

\subsubsection{Pixel検出器}
ここでは,ピクセル検出器をIBLと他3層に分けて説明する.ピクセル検出器は内部飛跡検出器の最内層に位置し,バレル部3層,エンドキャップ部は5枚のディスクからなり,それぞれに合計約1500個,約700個の検出器モジュールが配置されている.微小な読み出しチャンネルを2次元格子状に多数並べた作りをしているため,ピクセル検出器と呼ばれている.読み出しチャンネル毎のセンササイズが小さいため,位置分解能が高く,粒子密度の高い最内層でも粒子の飛跡の再構成の性能を維持する.\\
2014年にバレル部最内層で,ピクセルサイズが50 $\mathrm{\mu m^2}$であるIBLが導入された.残りのバレル部の3層はピクセルサイズが50 $\times$ 400 $\mathrm{\mu m^2}$である.ピクセルの読
バレル部とディスク部には,それぞれ同じ検出器が使用されており,検出器モジュールは長さ62.4 $\mathrm{m}$,幅21.4 $\mathrm{m}$であり,ピクセルセンサと読み出し用ASIC,信号処理用Flex基板からなる.センサはR- $\phi$方向に50 $\mathrm{\mu m}$,z方向に400 $\mathrm{\mu m}$に細分化されている.読み出しにはFEI3と呼ばれる18 $\times$ 160ピクセル分の読み出しチャンネルをもつASICが使用されている.センサは厚み250 $\mathrm{\mu m}$の

\subsubsection{Strip検出器}
SCTは,細長い短冊状の読み出しチャンネルを1次元方向に多数並べたストリップタイプのシリコン検出器である,ストリップ間隔は80 $\mathrm{\mu m}$,長さは128 $\mathrm{mm}$である.2枚のシリコンセンサを互いに40 $\mathrm{mrad}$の角度をつけて重ねて配置し,二次元位置情報を得る.SCTの読み出しチャンネルの総数は,約630万である.


\section{HL-LHCに伴うATLAS検出器のアップグレード}
LHCの長期シャットダウンとでLHCがHL-LHCへとアップグレードされる.本節では,HL-LHCのアップグレードに伴うATLAS検出器のアップグレードについて述べる.ATLAS検出器に求められるアップデート項目は以下である.
\begin{itemize}
\item 内部飛跡検出器の総入れ替え
\item トリガのアップグレード
\item カロリメータとミューオン検出器の読み出しシステムのアップグレード
\end{itemize}

\subsection{内部飛跡検出器のアップグレード}
以降,本論文に関わる内部飛跡検出器のアップグレードについて述べる.HL-LHCに向けて,内部飛跡検出器はInner Tracker(ITk)と呼ばれるシリコン検出器に置き換えられる.粒子密度の増加に対応できないためにTRT層は廃止され,内側にピクセル,それを覆うようにストリップ検出器が配置される.ピクセル検出器はバレル部とエンドキャップ部に5層,ストリップ検出器はバレル部に4層,エンドキャップ部に6層配置される予定である.

\subsubsection*{放射線耐性}       
ルミノシティの増加により,ITkは現行の内部飛跡検出器を上回る放射線耐性が要求される,内側2層のピクセル検出器には特に高い放射線耐性が要求され,センサにはプラナ型センサ,3Dセンサ,ダイアモンドセンサが候補として挙げられている.外側2層のピクセルセンサには放射線耐性があり,かつ低費用であるプラナ型センサが候補に挙げられている.

\subsubsection*{}

\subsection{HL-LHCに伴うITk開発}
ATLAS実験では,HL-LHCと呼ばれる加速器のアップグレードに伴い,ATLAS検出器の改良が求められる.本論文では
