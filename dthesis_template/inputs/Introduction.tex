\chapter{LHC ATLAS実験}
Large Hadron Collider(LHC)はスイス、ジュネーブにある欧州原子核研究機構(CERN)の地下100m,周長26.7kmのリングで構成される円形加速器である.最大で14TeVの重心系エネルギーで陽子陽子衝突させることが可能な,世界最大の陽子陽子衝突型加速器である.新粒子の探索や,ヒッグス粒子やトップクォーク等の質量が大きい粒子を多く生成できるので,結合定数などの精密測定も行うことが可能である.\\
LHCは2010年から運転を開始し,7TeVから8TeVの重心系エネルギーで2012年まで稼働した.この期間をLHC Run-1と呼び,瞬間最高ルミノシティは$0.77\times10^{34} \mathrm{cm^{-2}s^{-1}}$であった.その後,2013年から2015年までのシャットダウン期間で加速器のアップグレードを行い,2015年からは重心系エネルギー$13\mathrm{TeV}$でLHC Run-2が始まり,2018年まで続いた.Run-2の3年間で得られた積分ルミノシティは約$150\mathrm{fb}^{-1}$であった.\\
LHCは2年間のシャットダウン期間を経て2021年から重心系エネルギー$14\mathrm{TeV}$のLHC Run-3を予定している.Run-3が約3年間運転したのち,シャットダウン期間を挟んで,High Luminosity LHC(HL-LHC)が開始する予定である.

\section{ATLAS実験}
ATLAS実験はLHCの衝突点に設置されたATLAS検出器を用いて陽子陽子衝突から$\mathrm{TeV}$スケールまでの高エネルギー物理事象を探索する実験である.2012年には,LHC実験の1つであるCMS実験と共にヒッグス粒子を発見し,標準理論の完成お大きな役割を担った.世界最高エネルギーのLHCを使ったヒッグス粒子やトップクォークといった重い粒子の精密測定はATLAS実験の重要な目的の1つである.他にも超対称性粒子などの新粒子を発見することが特に大きな目的となっている.

\section{ATLAS検出器}
ATLAS検出器の全体図を示す.ATLAS検出器は高さ
This is the third page of the introductory chapter.
