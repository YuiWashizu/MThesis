\chapter*{Abstract}

Large Hadoron Collider (LHC) is a proton-proton collider type particle accelerator installed at the European Nuclear Research Organization (CERN). The ATLAS experiment is an experiment that uses the ATLAS detector installed at one of the four collision points of the LHC to accurately measure the standard model of elementary particles and search for physics beyond that. Aiming to start in 2026, the LHC High Luminosity LHC (HL-LHC) plan is underway. By increasing the luminosity of the LHC, it is possible to collide partons with large momentum frection of the proton, and this project is aimed at searching for heavy particles. In addition, due to the increase in statistics, it is expected that the sensitivity to new particles predicted by various models such as supersymmetry will increase. Along with the HL-LHC project, the inner track detector of the ATLAS detector will be replaced with a silicon detector called Inner Tracker (ITk) in order to cope with an increase in the amount of radiation received and an increase in the hit occupancy of the detector. The total replacement of is planned. For the total replacement the innter tracking detector, mass production of pixel sensor module is necessary. Mass-produced modules are subjected to various tests to achieve quality performance standards.\par
This thesis presents one of the test items, i.e. the response evaluation test against particle beams. This test has two methods: a method using self-trigger, which acquires data at the timing of the signal when a particle enters the module, and a method that uses a scintillator to generate data at the signal timing when the particle enters the scintillator. This is the method called an external trigger that performs acquisition. In this paper, we develop firmware and software necessary for establishing a response evaluation test method, acquire data using each of the two methods, and confirm and consider whether the response evaluation test can be actually performed. 
