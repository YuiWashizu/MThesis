\chapter*{Abstract}

Large Hadoron Collider (LHC) is a proton-proton collision type particle accelerator installed at the European Nuclear Research Organization (CERN). The ATLAS experiment is an experiment that uses the ATLAS detector installed at one of the four collision points of the LHC to accurately measure the standard model of elementary particles and search for physics beyond that. Aiming to start in 2026, the LHC High Luminosity LHC (HL-LHC) plan is underway. By improving the luminosity of the LHC, it is possible to collide with partons having large energies in protons, and this project is aimed at searching for heavy particles. In addition, due to the increase in statistics, it is expected that the sensitivity to new particles predicted by various models such as supersymmetry will increase. Along with the HL-LHC project, the internal track detector of the ATLAS detector has been replaced with a silicon detector called Inner Tracker (ITk) in order to cope with an increase in the amount of radiation received and an increase in the hit occupancy of the detector. The total replacement of is planned. For the total replacement, mass production of the pixel sensor module used inside is necessary. Mass-produced modules are subjected to various tests to achieve quality performance standards.\par
This paper deals with one of the test items, the response evaluation test for particle beams. This test has two methods: a method called self-trigger, which acquires data at the timing of the signal when a particle enters the module, and a method that uses a scintillator to generate data at the signal timing when the particle enters the scintillator. There is a method called an external trigger that performs acquisition. In this paper, we develop firmware and software necessary for establishing a response evaluation test method, acquire data using each of the two methods, and confirm and consider whether the response evaluation test can be actually performed. 
